% A workaround to allow relative paths in included subfiles
% that are to be compiled separately
% See https://tex.stackexchange.com/questions/153312/subfiles-inside-a-subfile-using-relative-paths
\providecommand{\main}{..}
\documentclass[\main/thesis.tex]{subfiles}

\begin{document}

\chapter{Uncertain Network Generator} \label{generator-chapter}

Since there are not many publicly available uncertain network datasets, to conduct experiments in following chapters, we put forward a way to generate uncertain networks based on deterministic networks. The way we generate uncertain networks is mainly based on three assumptions: (1) Edges that exist in deterministic networks tend to have high probability in corresponding uncertain networks; (2) In uncertain networks, except existential edges, there should exist some edges which do not exist in deterministic networks, and they tend to have low probability; (3) Based on a power law distribution, nodes with high degree are more likely to have new added edges. Based on these three assumptions, we generated the uncertain network by Algorithm \ref{Uncertain-Network-Generator}.

NEED MORE INTRODUCTION. SUCH AS NOISE PERCENTAGE.
\begin{algorithm}
  \KwData{A deterministic network $G$, non-existential edge percentage $p$}
  \KwResult{An uncertain network $\mathcal{G}$.}
  \For{each edge $e \in G.edges$}{
        Generate probability $P$ according to a Gaussian distribution with mean 0.8 and variance 1. (If not in the range (0,1], regenerate it.)\;
%         \While{probability $P$ not in the range (0,1]}{
%         	Regenerate probability $P$ according to a Gaussian distribution\;
%         }
        Assign probability $P$ to edge $e$\;
        Add edge $e$ to the uncertain network $\mathcal{G}$\;
    }
$NonExistentialEdgesCount\leftarrow|G.edges|\times p$\;
\While{$NonExistentialEdgesCount>0$}{
    Generate edge $e$ which is not in $\mathcal{G}.edges$\;
    Generate probability $P$ according to a Gaussian distribution with mean 0.2 and variance 1. (If not in the range (0,1], regenerate it.)\;
%         \While{probability $P$ not in the range (0,1]}{
%         	Regenerate probability $P$ according to a Gaussian distribution\;
%         }
    Assign probability $P$ to edge $e$\;
    Add edge $e$ to the uncertain network $\mathcal{G}$\;
    $NonExistentialEdgesCount \leftarrow NonExistentialEdgesCount-1$\;
}
\Return{uncertain network $\mathcal{G}$}
\caption{Uncertain Network Generator}
\label{Uncertain-Network-Generator}
\end{algorithm}

\end{document}