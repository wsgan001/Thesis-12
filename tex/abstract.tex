% A workaround to allow relative paths in included subfiles
% that are to be compiled separately
% See https://tex.stackexchange.com/questions/153312/subfiles-inside-a-subfile-using-relative-paths
\providecommand{\main}{..}
\documentclass[\main/thesis.tex]{subfiles}

\begin{document}

% environment for abstract.
\begin{abstract}
Many datasets can be represented as networks or graphs, where sets of nodes are joined together in pairs by links or edges. In the past, many works have been done on complex network analysis in deterministic graphs, graphs where the network structure is exactly and deterministically known. Recently, in many cases, uncertainty or imprecise information becomes a critical impediment to understanding and effectively utilizing the information contained in such graphs. There are many kinds of uncertainty in networks, such as edge uncertainty, node uncertainty, direction uncertainty and weight uncertainty. The problem of complex network analysis with uncertainty has become increasingly important. However, only a few studies take uncertainty into consideration.

In this thesis, we mainly focus on networks with edge uncertainty, which means the existence of some edges are uncertain. We propose efficient algorithms to solve problems such as entity ranking, link prediction and local community detection for networks with edge uncertainty. Due to the limited number of publicly available uncertain network datasets, we put forward a way to generate uncertain networks for evaluation purposes. Finally, we evaluate our algorithms using existing ground truth as well as based on common metrics to show the effectiveness of our proposed approaches.
\end{abstract}

\end{document}